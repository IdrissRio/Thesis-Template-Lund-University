
\newcommand{\paperIref}{%
Idriss Riouak, Christoph Reichenbach, G\"{o}rel Hedin and Niklas Fors.\\
``\textsc{A Precise Framework for Source-Level Control-Flow Analysis}''.\\
In \emph{21st International Working Conference on Source Code Analysis and Manipulation (SCAM),
pp. 1--11}. Virtual, 2021, IEEE.\\ Paper DOI: \href{https://doi.org/10.1109/SCAM52516.2021.00009}{10.1109/SCAM52516.2021.00009}.
}

\newcommand{\artifactIref}{%
\textbf{Paper~\ref{paper:paperOne}} introduces the \textsc{IntraCFG} and \textsc{IntraJ} static analyzer. The artifact includes the source code for both frameworks, along with the scripts used for the evaluation section presented in the paper. It was submitted to the ROSE (\textbf{R}ecognizing and Rewarding \textbf{O}pen Science in \textbf{SE}) festival and awarded the \href{https://icsme2021.github.io/cfp/AEandROSETrack.html}{``\emph{Open Research Objects}'' \includegraphics[height=10pt]{kappa/img/Open_Research.png}} and \href{https://icsme2021.github.io/cfp/AEandROSETrack.html}{``\emph{Research Objects Reviewed}'' \includegraphics[height=10pt]{kappa/img/Research_Objects.png}} badges.\\
DOI: \href{https://doi.org/10.5281/zenodo.5296618}{10.5281/zenodo.5296618}.
}


\newcommand{\paperIIref}{%
Idriss Riouak, Niklas Fors, G\"{o}rel Hedin, and Christoph Reichenbach.\\
``\textsc{IntraJ: An On-Demand Framework for Intraprocedural Java Code Analysis}''.\\
Submitted for publication.
}











% From here on, the macros.tex file contains settings about the layout of the
% Thesis itself. Do not change anything below this line unless you know what
% you are doing.


% Defines the layout settings for the thesis
\newcommand{\pageSettings}{
 \setlength{\voffset}{3pt}
 \setlength{\hoffset}{-1in}
 \setlength{\textheight}{520pt}
 \setlength{\textwidth}{327pt}
 \setlength{\oddsidemargin}{93pt}
 \setlength{\evensidemargin}{195pt}
\setlength{\marginparsep}{0pt}
}


% ------------------ Change Chapter title row layout
\makeatletter
\def\thickhrulefill{\leavevmode \leaders \hrule height 1ex \hfill \kern \z@}
\def\@makechapterhead#1{%
  \markmargin%
  \addtolength{\spiff}{3cm}%
  \vspace*{10\p@}%
  {\parindent \z@ \raggedleft \reset@font
            \sffamily \bfseries \scshape %\@chapapp{} \thechapter%
        \par\nobreak%
        \interlinepenalty\@M%
    \Huge #1\par\nobreak
    \hrulefill
    \par\nobreak
    \vskip 80\p@
  }}
\def\@makeschapterhead#1{%
  \vspace*{10\p@}%
  {\parindent \z@ \raggedleft \reset@font
            \sffamily \bfseries \scshape \vphantom{\@chapapp{} \thechapter}
        \par\nobreak
        \interlinepenalty\@M
    \Huge  #1\par\nobreak
    \hrulefill
    \par\nobreak
    \vskip 80\p@
  }}

%------------------------ part page layout
%% from book.cls:
\renewcommand\part{%
   \if@openright
      \cleardoublepage
   \else
      \clearpage
   \fi
   \thispagestyle{plain}%
  \if@twocolumn
    \onecolumn
    \@tempswatrue
  \else
    \@tempswafalse
  \fi
  \null\vfil
  \secdef\@part\@spart}

\def\@part[#1]#2{%
    \ifnum \c@secnumdepth >-2\relax
      \refstepcounter{part}%
      \addcontentsline{toc}{part}{#1}%
    \else
      \addcontentsline{toc}{part}{#1}%
    \fi
    \markboth{}{}%
    {\centering
     \interlinepenalty \@M
     \normalfont
     \Huge \reset@font \hrulefill \\ \raggedleft \vspace{5mm} \sffamily%
            \bfseries \scshape {#2} \hspace{15mm}\ \\ \hrulefill \par}%
    \@endpart}
\def\@spart#1{%
    {\centering
     \interlinepenalty \@M
     \normalfont
     \Huge \sffamily \bfseries scshape #1\par}%
    \@endpart}
\def\@endpart{\vfil    %% \newpage -- HWL: no pagebreak with part env
              \if@twoside
                \null
                \thispagestyle{empty}%
                %%\newpage
              \fi
              \if@tempswa
                \twocolumn
              \fi}
\makeatother
%-----------------------------------------------



%Make fancy header and no footer
\pagestyle{fancy}
\renewcommand{\chaptermark}[1]{\markboth{#1}{}}
\renewcommand{\sectionmark}[1]{\markright{\thesection\ #1}}
\fancyhf{}
\fancyhead[LE, RO]{\bfseries\thepage}
\fancyhead[LO]{\itshape \rightmark}
\fancyhead[RE]{\truncate{.95\headwidth}{\itshape \leftmark}}
\renewcommand{\headrulewidth}{0.5pt}
\renewcommand{\footrulewidth}{0pt}%
\setlength{\headheight}{75pt}
\fancypagestyle{plain}{
  \fancyfoot[LE, RO]{\chapterFoot}
  \fancyhead{}
  \renewcommand{\headrulewidth}{0pt}
}

\newcommand{\chapterFoot}{}
\newcommand{\markmargin}{}
\newlength{\spiff}

\pageSettings % -------------------------> Applying the style to the thesis. See macros.tex for more details
\allsectionsfont{\sffamily} % -----------> All sections in sans serif


\newtheorem{definition}{Definition}
\newcommand{\Chapter}[1]{Chapter~\ref{#1}}


\newfloat{paperfoot}{b}{paper}
\newcommand{\paperRemark}[1]{
  \begin{paperfoot}%
  \hrulefill \flushleft \footnotesize #1 \end{paperfoot}
}





